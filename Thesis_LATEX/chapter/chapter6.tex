\chapter{总结与展望}

\section{总结}
随着电子以及计算机科学技术的高速发展,视频智能视频监控领域得到了长 足的发展,各式各样的视频监控产品早已融入到了大众的生活中。但是,目前大 部分投入应用的视频监控系统还停留在比较基础的阶段,功能较为单一,系统的 主要流程为视频获取、传输以及录制,因此,这些系统就需要有专门的人员在一 旁值班留守,这样就很容易造成重要信息的丢失。特别是当出现特殊紧急情况时, 需要抽出人力对大量存储在硬盘上的录像进行排查。针对这些问题,本文展开了 弱小目标检测及其智能分析实时系统的研究和开发工作,了解了目前国内外目标 检测系统的发展情况,同时针对小目标检测这一目前技术难点对检测系统提出 进,并在当下流行的硬件平台上实现了本文提出的Atrous Faster R-CNN针对弱小目标的实时检测;同时在主机端实现了对运动目标的智能识别,判断其是否为感兴趣目标。与传统的行人识别方法相比,本文方法极大提高了算法的效率。通过实 际测试验证可见,该检测系统能够良好工作且可以应用到实际环境中。

本文的创新性主要体现在以下几个方面:
(1)研究基于深度的目标检测算法,并提出Atrous Faster R-CNN,该算法的在VOC 2007上经过测试比当前流行的Faster R-CNN算法检测精度高,尤其对于小目标,显著提升了其检测精度,满足了实际视频监控系统的实际需求;
(2)将基于深度学习的目标检测算法与在线学习的目标跟踪算法结合起来,实现目标的自动检测与长时间跟踪,以及跟丢后的重新初始化等实际问题
(3)采用CPU和GPU协同工作的方式将目标检测算法移植到嵌入式平台,探索深度学习在嵌入式上运行这一未来趋势。

\section{展望}
作为多媒体处理领域里的热门研究方向之一,视频智能监控的相关技术不仅 吸引了国内外众多学者的关注,同时还有一大批工程人员将其实现并应用到实际 生活中。而作为其中发展最快的目标检测及分析技术,更是渗透到了生活中的方方面面。本文所设计和实现的算法能够对小目标实现检测以及长时间的跟踪,检测效果良好,准确率较高,但同时也还有很多可以进一步完善的地方。基于深度学习的目标检测计算技术显著提高了精度,目前已经进入实际应用阶段,但是深度神经网络计算所需的存储与计算资源太大,限制它在嵌入式平台上的应用。本文只是简单探索了在NVIDIA嵌入式GPU上的移植,而未来要做的工作包括对深度模型进行裁剪,实现精度与资源消耗的权衡,即用更小的模型来完成目标检测,虽然可能会损失小部分精度。以及现在正处于研究的深度模型压缩和模型量化等方向,都为深度学习在嵌入式平台上运行提供了另外的解决思路。