
\chapter*{致谢}
\addcontentsline{toc}{chapter}{\numberline{} 致谢}
%%%\addtocontents{toc}{致谢}
时光如白驹过隙,两年半的研究生生涯已经接近了尾声,在北理六年半的生活也要宣告结束。在研究生期间,我遇到了许多帮助我、支持我的人。对此,我把自己的感激之情在这论文最后记录下来,以示纪念。
研究生的学习是实践型的学习,若无老师和同学的指导帮助,想必道路上会有许多绊脚石,让我难以前行。
首先,感谢我的指导导师陈文颉副教授,他对我们温柔体贴,悉心指导。若无他高瞻远瞩的指路,我的研究方向就会与科研的潮流脱轨。若无他细心的指导,我将无法顺利完成硕士学位的攻读。在这篇论文完成之际,我谨以此向我的导师表达深深的谢意。
感谢模式识别与智能控制实验室的各位老师们:陈杰教授、窦丽华教授、孙健教授、方浩教授、彭志红教授、辛斌副教授、邓方副教授、白永强副教授、张佳老师等,是他们在学习和生活中给我们提出富有学术水平和生活经验的建议,为我在前进的道路上点亮一盏明灯。
另外,实验室和寝室的同学也是我克服困难,努力前进的动力。712实验室的高慧琳学姐、朱皓学长、吴晓文学长、周峰宜学长、肖驰学长、郝克学长、黄新宇学长、赵栋学长、王茂章学长、侯棋文、田亚、高振巍、秦莹莹、李东轩、塞音、游清与我同甘共苦,共同克服科研难关。感谢成艺丰、杜晓龙、丁东升,他们一起与我生活与我科研。
到了即将天南地北,走上各自工作岗位的时候,祝我的小伙伴们学习、工作顺利,生活愉快。
感谢我的父母,他们就是我坚强的后盾。他们十几年如一日的支持我,才能让我顺利攻读硕士学位。
最后,感谢评阅的老师、专家、学者、感谢你们抽出宝贵的时间,祝你们工作顺利,万事安康。

\ \ \ \ \ \ \ \ \ \ \ \ \ \ \ \ \ \ \ \  \ \ \ \ \ \ \ \ \ \ \ \ \ \ \ \ \ \ \ \   \ \ \ \ \ \ \ \ \ \ \ \ \ \ \ \ \ \ \ \ \ \ \ \ \ \ \ \ \ \ \  \ \ \ \ \ \ \ \ \ \  \ \ 北京理工大学6号楼712

\ \ \ \ \ \ \ \ \ \ \ \ \ \ \ \ \ \ \ \  \ \ \ \ \ \ \ \ \ \  \ \ \ \ \ \ \ \ \ \  \ \ \ \ \ \ \ \ \ \ \ \ \ \ \ \ \ \ \ \  \ \ \ \ \ \ \ \ \ \  \ \ \ \  \ \ \   \ \ \ \ \ \ 2016年12月4日

