
\chapter{绪论}
本课题针对周视目标监控与跟踪这一应用场景,研究其中关键的技术点:目标检测与长时间目标跟踪问题。

\section{研究背景与意义}
目标检测与跟踪是近年来计算机视觉领域中备受关注的前沿方向,它从图像序列中检测并跟踪特定目标,并对其行为进行理解和描述。基于视觉的自动目标检测与跟踪系统是在不需要人工干预的情况下,运用计算机视觉技术对摄像头拍摄的图像序列进行处理,实现对感兴趣目标的检测与跟踪,并在此基础上对目标的行为进行分析和预测,最终既可以完成日常的管理工作又可以在异常情况出现的时候及时做出响应。
目标检测与跟踪在近二十年来一直是比较热的研究课题,主要原因在于它广泛地应用于军事、民用领域:
\begin{namelist}{}
	\item(1)武器装备。在军事领域,目标检测与跟踪已经大量应用于战场侦察、武器控制等方面,在一定程度上提高了武器装备的智能化水平。典型的应用有日本90式主战坦克的火控系统,美国海军陆战队的AV-8A系统。此外,很多先进的导弹系统都采用目标检测与跟踪技术,如欧洲最新型的近程空对空导弹。在航天领域,我国载人航天飞船“神州七号”于08年9月28日顺利完成太空作业任务,在返回舱降落到距离地面不到一万米的时候,执行任务的代号为“雄鹰”的长机正是通过红外与可见光跟踪器对返回舱进行精确跟踪,才确保返回舱的安全降落。
	\item(2)安防。目标检测与跟踪技术在智能视频监控系统中的应用非常广泛,最常见的是对民宅、停车场、银行等公共场合的监控,以防止恐怖袭击、偷盗等破坏行为的发生,保障社会安全。由欧盟出资启动的机场智能监控项目AVITRACK系统,能够对停机坪场景进行目标跟踪和异常行为检测与报警,为机场安防这一911 后重大的安全课题提供了智能化的解决方案 \cite{avitrack}。
	\item(3)行人保护。据资料显示 \cite{vehicle},当车祸发生时如果机动车辆的速度不超过30km/h,行人的死亡率大约为10\%,然而当机动车辆的速度超过45km/h的时候,行人的死亡率将会超过50\%。行人保护系统通过车载摄像头实时检测并跟踪行人,从而获取行人的位置、速度等信息,并主动采取安全措施,例如刹车或释放气囊,进行行人保护。目前国外著名汽车厂商沃尔沃、奔驰等都在进行这方面的研究与尝试。
\end{namelist}

除了上述几个方面的应用之外,目标检测与跟踪也是人际交互、虚拟现实、3D重构等领域的核心技术之一。
随着自动目标检测与跟踪系统应用地越来越广泛,应用场景也变得越来越复杂,各个应用场景之间的成像环境以及上下文信息也各不相同,这对目标的检测、跟踪又提出了新的挑战。根据摄像头是否运动,应用场景可以分为静态场景和动态场景。静态场景下最典型的应用就是智能视频监控。在动态场景下,摄像头与目标之间会发生相对平移与旋转,导致目标的成像产生较大的尺度与姿态变化,加大了目标检测与跟踪的难度。然而动态场景下的目标检测与跟踪在军事和民用上都有着很大的应用价值。本课题来源于某科研项目,主要研究以车载平台为代表的动态场景下的自动目标检测与跟踪,实时解算出目标在图像场景中的精确位置,并输出目标偏离系统视轴的方位和俯仰误差信号,通过伺服控制回路,驱动云台跟踪目标。
自动目标跟踪系统主要涉及目标检测(object detection)和目标跟踪(target tracking)两大技术:其中检测器(detector)用于初始化跟踪器(tracker)的输入;并且在之后的跟踪过程中,跟踪结果很可能会出现漂移,此时检测器又可以用于重新初始化目标位置,消除累积误差;除此之外,当下流行的tracking-by-detection跟踪算法也是基于检测的。因此,检测器的有效性是整个系统能够长时间工作的关键,例如流行的TLD跟踪框架 \cite{tld} 正是凭借着检测器的不断更新才保证了跟踪器长时间的稳健跟踪。本课题的一大重点便是研究车载场景下目标的实时检测,检测器的优劣直接决定了整个系统的性能。在检测器定位到目标的初始位置后,跟踪器在接下来的图像序列中确定目标的位置,一方面能够剔除错误的检测结果,另一方面能够预测目标的位置和方向,甚至可以完成更高层次的任务,例如目标的行为理解。

\section{研究现状}
国内外学者对运动目标的检测与跟踪进行了大量的研究,每年都有很多优秀的研究成果出现,但目标本身的不确定性和所处环境的复杂性造成绝大多数目标检测与跟踪算法只适用于某一特定环境,或只能检测跟踪某一类特定目标。
\subsection{目标检测技术研究现状}
目标检测是大量高级视觉任务的必备前提,包括活动或事件识别、场景内容理解等。而且目标检测也被应用到很多实际任务,例如智能视频监控,基于内容的图像检索,机器人导航和增强现实等。目标检测的研究具有重要意义,在过去几十年里吸引了大批研究人员的密切关注。随着机器学习理论和特征分析技术的发展,近十几年目标检测取得了长足的发展。尽管如此,当前方法的检测准确率仍然较低而不能应用于实际通用的检测任务。目标检测主要包含两类不同的检测任务:目标实例检测和目标类别检测。本课题主要关心目标类别检测。
特殊类别的目标检测,例如人脸和行人,检测技术已经较为成熟。Viola \cite{via-jones-face} 提出基于AdaBoost算法框架,使用haar-like特征分类,然后采用滑动窗口搜索策略实现准确有效的定位。这是第一种能实时处理并给出很好检测率的类别检测算法,主要应用于人脸检测。Dalal \cite{hog} 提出使用图像局部直方图(HOG)作为特征,利用支持向量机(SVM)作为分类器进行行人检测,取得了较大的成功。更为普遍的目标检测工作关注自然图像中一般类别的检测,大部分类别都能够发生非刚性形变,为此Felzenszwalb\cite{dpm} 提出了目标类别检测领域最具影响力的方法之一,多尺度形变部件模型(DPM),继承了HOG特征和SVM分类器的优点。DPM目标检测器由一个根滤波器和一些部件滤波器组成,组件间的形变通过隐含变量进行推理。由于目标模板分辨率固定,算法采用滑动窗口策略在不同尺度和宽高比图像上搜索目标。后续工作采用不同策略加速了DPM的穷尽搜索策略。Malisiewicz \cite{ensemble} 提出一种简单高效的集成学习算法用于目标类别检测,该方法分别为每个正样本训练一个使用HOG特征的线性SVM,通过集成每个样本的线性SVM结果达到优良的泛化性能。Ren \cite{ren} 认为之前基于HOG特征的检测算法中HOG特征是人为设计的,判别能力较弱且不直观。为此提出一种基于稀疏表达学习理论的稀疏编码直方图特征(HSC),并使用HSC代替DPM目标检测算法中的HOG特征,检测率高于原方法。Wang \cite{wang} 为去除DPM模型需要人为制定组件个数及组件间关系的穷尽搜索的限制,提出了一种新的特征表达方式Regionlets,采用选择性搜索策略对每个候选检测区域进行多种区域特征的集成级联式分类。Regionlets保留了目标的空间结构关系,灵活地描述目标,包括发生形变的目标。2012年前,目标检测中分类任务的框架就是使用人为设计的特征训练浅层分类器完成分类任务,最佳算法就是基于DPM框架的各种改进算法。2012年,Krizhevsky \cite{alexnet} 提出基于深度学习理论的深度卷积神经网络(DCNN)的图像分类算法,使图像分类的准确率大幅提升,同时也带动了目标检测准确率的提升。Szegedy \cite{multibox} 将目标检测问题看作目标mask的回归问题,使用DCNN作为回归器来预测输入图像中候选区域的mask。Sermanet \cite{overfeat} 提出一种DCNN框架OverFeat,集成了识别、定位和检测任务,为分类训练一个CNN,为每个类训练一个定位用的CNN。OverFeat对输入图像采用滑动窗口策略用分类模型确定每个窗口中目标的类别,然后使用对应类别的定位模型预测目标的bounding box,根据分类分数为每个类选出候选bounding box进行合并,得到最终的检测结果。与OverFeat不同,R-CNN \cite{rcnn} 采用选择性搜索策略而不是滑动窗口来提高检测效率。R-CNN利用选择性搜索方法在输入图像上选择若干候选bounding box,对每个bounding box利用CNN提取特征,输入到为每个类训练好的SVM分类器,得到bounding box属于每个类别的分数。最后,R-CNN使用非极大值抑制方法(NMS)舍弃部分bounding box,得到检测结果。DCNN的基本结构源自Krizhevsky的7层网络结构设计,为了提高DCNN的分类和检测准确率,Simonyan和Szegedy \cite{vgg} 设计了层数为22层的深度卷积神经网络,采用的检测框架都类似R-CNN。目前,深度卷积神经网络是多个目标类别检测数据集上的state of the art。RCNN存在着重复计算的问题(proposal的region有几千个,多数都是互相重叠,重叠部分会被多次重复提取feature),于是Ross B. Girshick借鉴Kaiming He的SPP-net \cite{sppnet} 的思路单枪匹马搞出了Fast-RCNN \cite{fast-rcnn},跟RCNN最大区别就是Fast-RCNN将proposal的region映射到CNN的最后一层conv layer的feature map上,这样一张图片只需要提取一次feature,大大提高了速度,也由于流程的整合以及其他原因,在VOC2007上的mAP也提高到了68\%。探索是无止境的。Fast-RCNN的速度瓶颈在region proposal上,于是Ren等 \cite{faster rcnn}将Region proposal也交给CNN来做,提出了Faster-RCNN。Fater-RCNN中的region proposal netwrok实质是一个Fast-RCNN,这个Fast-RCNN输入的region proposal的是固定的(把一张图片划分成$n\times n$个区域,每个区域给出9个不同长宽比和scale的proposal),输出的是对输入的固定proposal是属于背景还是前景的判断和对齐位置的修正(regression)。Region proposal network的输出再输入第二个Fast-RCNN做更精细的分类和bounding box的位置修正。Fater-RCNN速度更快了,而且用VGG net作为feature extractor时在VOC2007上mAP能到73\%。
\subsection{目标跟踪技术研究现状}
目标跟踪就是在目标检测的基础上,通过对视频序列的分析,得到目标在每帧图像中的位置信息。动态场景中的目标跟踪与静态场景中的目标跟踪相比,更为复杂与困难,它是计算机视觉领域中的一个非常具有挑战性的课题。经过研究者们多年的研究,目前已经提出了一些比较成熟的跟踪算法,主要有基于特征点的跟踪算法、基于区域的跟踪算法、基于动态轮廓的跟踪算法和基于运动估计的跟踪算法等。

基于特征点的跟踪算法 \cite{keypoint} 的基本思想是采用一些特征点来描述目标,通过跟踪这些特征点,达到跟踪目标的目的,主要涉及特征点提取和特征点匹配技术。常用的特征点有Harris \cite{harris} 特征点、KLT \cite{klt} 特征点、SIFT特征点和SURF特征点等。特征点匹配的过程主要是求取最优匹配点对。在这类算法中,当目标发生旋转、形变或遮挡时,跟踪的关键在于提取的特征点是不是具有鲁棒性,因此提取合适的特征点至关重要。

基于区域的跟踪算法 \cite{correlation} \cite{error} 采用bounding box来描述目标。这类方法一般通过提取目标区域的各种特征,如亮度、颜色、纹理或光流场等,并将不同时刻的目标区域通过特征匹配进行对应。相关跟踪是最典型的区域跟踪方法,它往往使用固定模板,因而在较短的时间范围内是可靠的,但是难以适应长时间跟踪,因为目标外观会随时间发生较大变化;此外,区域跟踪方法还包括基于局部或全局统计信息的方法,如全局或局部颜色直方图、纹理直方图等,这些方法在目标形变和遮挡的情况下能得到较好的跟踪效果,缺点是这种模型的表达性较差,易于受到外界干扰。
基于动态轮廓的跟踪算法是根据目标的边界轮廓来进行跟踪的。该算法对目标的描述比较简单有效,对存在目标部分遮挡的情况同样适用。但它依赖于目标初始轮廓的精确描述,一旦初始化结果不准确或失败,将导致后续跟踪的失败。基于运动估计的跟踪算法通过对目标的运动状态进行估计来达到跟踪目标的目的。这类方法包括了传统光流法、金字塔Lucas-Kanade 光流跟踪算法、卡尔曼滤波算法、扩展卡尔曼滤波算法、粒子滤波算法等等,它们各有优劣。

目前跟踪算法可以被分为产生式(generative model)和判别式(discriminative model)两大类别。

产生式方法运用生成模型描述目标的表观特征,之后通过搜索候选目标来最小化重构误差。比较有代表性的算法有稀疏编码(sparse coding),在线密度估计(online density estimation)和主成分分析(PCA)等。产生式方法着眼于对目标本身的刻画,忽略背景信息,在目标自身变化剧烈或者被遮挡时容易产生漂移。

与之相对的,判别式方法通过训练分类器来区分目标和背景。这种方法也常被称为tracking-by-detection。近年来,各种机器学习算法被应用在判别式方法上,其中比较有代表性的有多示例学习方法(multiple instance learning), boosting和结构SVM(structured SVM)等。判别式方法因为显著区分背景和前景的信息,表现更为鲁棒,逐渐在目标跟踪领域占据主流地位。值得一提的是,目前大部分深度学习目标跟踪方法也归属于判别式框架。

近年来,基于相关滤波(correlation filter)的跟踪方法因为速度快,效果好吸引了众多研究者的目光。相关滤波器通过将输入特征回归为目标高斯分布来训练 filters。并在后续跟踪中寻找预测分布中的响应峰值来定位目标的位置。相关滤波器在运算中巧妙应用快速傅立叶变换获得了大幅度速度提升。目前基于相关滤波的拓展方法也有很多,包括核化相关滤波器(kernelized correlation filter, KCF), 加尺度估计的相关滤波器(DSST)等。
此外,动目标跟踪方法也可分为基于二维和基于三维的跟踪方法。基于二维的方法实现简单,但难以解决遮挡、复杂背景干扰、关照变化等问题,而基于三维建模的方法由于引入了目标物体的三维空间信息,比基于二维的方法更具鲁棒性,但是相应的研究难度也更大。
