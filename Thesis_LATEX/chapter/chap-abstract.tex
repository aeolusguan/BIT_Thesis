\newpage
\pagenumbering{Roman}

%%%%%%%%%%%%%%%%%%%%%%%%%%%%%%%%%%%%%%%%%%%%%%%%%%%%%%%%%%%%%%%%%%%%%%%%%%%%%%%%%%
%                                中文摘要
%%%%%%%%%%%%%%%%%%%%%%%%%%%%%%%%%%%%%%%%%%%%%%%%%%%%%%%%%%%%%%%%%%%%%%%%%%%%%%%%%%


\chapter*{摘要}

随着图像处理、计算机视觉和模式识别的快速发展,目标检测、跟踪和识别技术已经广泛应用于实际生活中。鲁棒性强的检测识别技术更是所有检测系统追求的目标,尤其是对视野中经常会出现的小尺度目标,都希望系统能够拥有很低的误检率和良好的识别率。要达到这个目的,就需要充分考虑系统所应用的环境,从而提出性能优良的检测识别算法。
本文采用基于region proposal深度学习的目标检测框架,分析了现有方法的优势与劣势,发现现有算法都不能对小目标较好地检测。为此,本文提出了在Faster R-CNN这一流行的深度学习目标检测算法的基础上提出了Atrous Faster R-CNN目标检测算法,相比于Faster R-CNN提高的整体的检测精度,对于小目标的检测精度更是取得了10\%以上的提升。本文采用P-N learning的方式结合基于深度学习的目标检测算法与在线学习的目标跟踪器实现了对目标的自动检测以及长时间跟踪,并能够在目标跟丢的情况下重新初始化。最后本文利用CUDA 并 行编程模型,在 NVIDA 公司最新推出的嵌入式 GPU 平台 Jetson TK1 上实现目标检测算法,以探索在嵌入式平台运行深度学习视觉算法这一未来趋势。

该目标检测系统的整体测试可以表明,该系统具有较强的鲁棒性,能够克服背景变化明显和光照变化快速等条件的影响,无论是静止还是运动的小目标,该系统都能保持良好的性能。本文提出的Atrous Faster R-CNN算法在NVIDIA Titan X上的运行速度为220ms左右,经过对输入图像的降采样以及适当的压缩,在其纳入是TK1上的运行时间为300ms每桢。
\vskip 0.3cm \textbf{关键词}: 深度学习,目标检测,长时间目标跟踪 嵌入式GPU
%%%%%%%%%%%%%%%%%%%%%%%%%%%%%%%%%%%%%%%%%%%%%%%%%%%%%%%%%%%%%%%%%%%%%%%%%%%%%%%%%%
%                                英文摘要
%%%%%%%%%%%%%%%%%%%%%%%%%%%%%%%%%%%%%%%%%%%%%%%%%%%%%%%%%%%%%%%%%%%%%%%%%%%%%%%%%%
\newpage
\chapter*{Abstract}
With the rapid development of image processing, computer vision and pattern recognition, target detection, tracking and recognition technology has been widely used in real life. Robust detection and recognition technology is the goal of all detection systems, especially for small-scale targets often appear in the field of vision, hope the system can have a very low false detection rate and good recognition rate. To achieve this goal, we need to give full consideration to the environment used by the system, thus proposed the performance of the detection algorithm. In this paper, the target detection framework based on region proposal deep learning is used to analyze the advantages and disadvantages of the existing methods. It is found that the existing algorithms can not detect small targets better. To solve this problem, this paper proposes an algorithm of target detection for Atrous Faster R-CNN, which is based on Faster R-CNN. Compared with Faster R-CNN, the detection accuracy is improved. The target detection accuracy is achieved more than 10\% incresement. In this paper, P-N learning is combined with the target detection algorithm based on depth learning and the target tracker of on-line learning to realize automatic detection and long time tracking of the target, and can be re-initialized in the case of target follow-up. In the end, this paper uses the CUDA parallel programming model to implement the target detection algorithm on Jetson TK1, the latest embedded GPU platform of NVIDA, to explore the future trend of depth learning visual algorithm in embedded platform.

The overall test of the target detection system can show that the system has strong robustness and can overcome the influence of the background change and the rapid change of the illumination conditions. The system can keep good whether it is stationary or moving small targets performance. In this paper, the Atrous Faster R-CNN algorithm runs on the NVIDIA Titan X at about 220 ms. After the input image is down-sampled and compressed properly, the run-time on the TK1 is 300 ms per frame.
\textbf{Keywords}: deep learning, object detection, long term tracking, embeded GPU