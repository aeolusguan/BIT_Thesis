\newpage
\pagenumbering{Roman}

%%%%%%%%%%%%%%%%%%%%%%%%%%%%%%%%%%%%%%%%%%%%%%%%%%%%%%%%%%%%%%%%%%%%%%%%%%%%%%%%%%
%                                中文摘要
%%%%%%%%%%%%%%%%%%%%%%%%%%%%%%%%%%%%%%%%%%%%%%%%%%%%%%%%%%%%%%%%%%%%%%%%%%%%%%%%%%


\chapter*{摘要}

\hspace*{0.85cm}随着图像处理、计算机视觉和模式识别的快速发展,目标检测、跟踪和识别技术已经被广泛应用于实际生活中。鲁棒性强的检测识别技术是所有自动目标搜索系统追求的目标,尤其是对视野中经常会出现的小尺度目标,也希望系统能够拥有很低的误检率和良好的识别率。另一方面长时间的目标跟踪也是自动目标跟踪系统急需解决的问题。本文针对自动目标搜索与跟踪系统这一研究课题主要研究如下所述:

近年来,随着深度学习技术的兴起,目标检测技术取得了重大的突破,以R-CNN为代表的基于region proposal的深度学习目标检测算法显著提升了目标检测的精度,以SSD(Single Shot Detection)为代表的基于回归的检测算法已经能够在服务器端实时地进行检测。但是现有方法对于小尺度目标均不能很好地进行检测,而小尺度目标在实际的目标搜索系统中却很常见。为此,本文在Faster R-CNN这一流行的深度学习目标检测算法的基础上提出了Atrous Faster R-CNN目标检测算法,相比于Faster R-CNN提高了整体的检测精度,对于小尺度的目标检测精度更是取得了10\%以上的提升。

很多实际应用场景需要将目标检测器运行在嵌入式环境下,例如家庭服务机器人、小型民用无人机等。为此本文将Atrous Faster R-CNN网络结构进行裁剪以适应嵌入式设备的计算和存储资源,并将其移植到Jetson TX1平台上。为了进一步提升精简版的Atrous Faster R-CNN在嵌入式设备上的检测精度,本文采用了两个工程上的trick:与ground-truth重叠度在某个模棱两可的范围内的proposal只进行坐标回归不进行分类,难分样本挖掘。

本文采用P-N learning的方式结合基于深度学习的目标检测算法与在线学习的目标跟踪器实现了对目标的自动检测以及长时间跟踪,并能够在目标丢失的情况下重新初始化。基于region proposal的深度学习目标检测模块具有较高的准确性,为在线学习的分类器提供额外的正负样本,P-N learning在理论上保证只要目标检测器的精度高于50\%,它提供的正负样本就会对在线学习的分类器的性能有所改善。本文以Struck算法为例介绍了如何将目标检测器嵌入到在线式跟踪器中,并克服了Struck跟踪算法只能跟踪单一尺度目标的问题。

\vskip 0.3cm \textbf{关键词}: 深度学习;目标检测;长时间目标跟踪;嵌入式GPU
%%%%%%%%%%%%%%%%%%%%%%%%%%%%%%%%%%%%%%%%%%%%%%%%%%%%%%%%%%%%%%%%%%%%%%%%%%%%%%%%%%
%                                英文摘要
%%%%%%%%%%%%%%%%%%%%%%%%%%%%%%%%%%%%%%%%%%%%%%%%%%%%%%%%%%%%%%%%%%%%%%%%%%%%%%%%%%
\newpage
\chapter*{Abstract}
\hspace*{0.85cm}With the rapid development of image processing, computer vision and pattern recognition, target detection, tracking and recognition technology has been widely used in real life. Robust detection and recognition technology is the goal of all detection systems, especially for small-scale targets often appearing in the field of vision, hoping the system can have a very low false detection rate and good recognition rate. To achieve this goal, we need to give full consideration to the environment that the system applicated in, so that proposed the performance of the detection algorithm. In this paper, the target detection framework based on region proposal deep learning is used to analyze the advantages and disadvantages of the existing methods. It is found that the existing algorithms can not detect small targets very well. To solve this problem, this paper proposes an algorithm of target detection for Atrous Faster R-CNN, which is based on Faster R-CNN. Compared with Faster R-CNN, the detection accuracy is improved. The target detection accuracy is achieved more than 10\% incresement. 

Many applications need to run object detection in embeded platform, for home-service robots and UAV. To this end, this paper bases the Atrous Faster R-CNN on a tiny model to fit the computation and memory resources of embeded platform and ports the tiny Atrous Faster R-CNN to the Jetson TX1 platform. Futhermore, we advocate two tricks to improve the detection accuracy of the tiny model: perform only bounding box regression for regions that partly overlap the ground-truth, and the hard example mining.

In this paper, P-N learning is combined with the target detection algorithm based on depth learning and the target tracker of on-line learning to realize automatic detection and long time tracking of the target, and can be re-initialized in the case of target follow-up. We take the Struck algorithm for example and explains how to embed the detection module to tracking context to achieve long-term tracking. And the proposed method overcomes the issue that Struck can only track objects at a single scale. 

\textbf{Keywords}: deep learning; object detection; long term tracking; embeded GPU