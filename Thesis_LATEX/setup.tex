%%[mhb] 05/02/10: v0.1 for BIT PhD thesis

\makeatletter


%%参考文献GBT7714-2005NLang相关宏包
\usepackage{pdflscape}
\usepackage{array}
\newcommand{\PreserveBackslash}[1]{\let\temp=\\#1\let\\=\temp}
\newcolumntype{C}[1]{>{\PreserveBackslash\centering}p{#1}}
\newcolumntype{R}[1]{>{\PreserveBackslash\raggedleft}p{#1}}
\newcolumntype{L}[1]{>{\PreserveBackslash\raggedright}p{#1}}
\usepackage{enumerate}
\usepackage[numbers,square,sort]{natbib} %数字,方括号,上标,排序与压缩
\usepackage{hypernat}
%%调入一些宏包
\usepackage{amssymb, amsmath,color}
\usepackage{fancyhdr}
\usepackage{fancyref}
\usepackage{mathrsfs}
\usepackage{subfigure}
\usepackage{listings,color}
\DeclareMathAlphabet{\mathpzc}{OT1}{pzc}{m}{it}


%%页面大小设置
%\usepackage[top=3.5cm,bottom=3cm,left=2.7cm,right=2.7cm,headheight=1cm,footskip=0.7cm]{geometry}
\usepackage[top=3.7cm,bottom=3.1cm,left=2.7cm,right=2.7cm,headheight=1cm,footskip=1.3cm]{geometry}

%%页眉页脚
\pagestyle{plain}
\fancypagestyle{plain}{%
	\fancyhf{} % 清空当前设置
	\renewcommand{\headrulewidth}{0.5pt} % 页眉与正文之间的水平线粗细
	\renewcommand{\footrulewidth}{0pt}
	\lhead{}
	\chead{\zihao{5}\songti 北京理工大学硕士学位论文}
	\rhead{}
	\lfoot{}
	\cfoot{\thepage}
	\rfoot{}
}


%%调入图表相关宏包
\usepackage{graphicx}
\usepackage{threeparttable}
\usepackage{dcolumn}
\usepackage{multirow}
\usepackage{booktabs}
\usepackage{listings}
\usepackage{booktabs}
\usepackage{tabu}
\usepackage{rotating}
\usepackage{subfigure}
%===设置插图路径!强烈建议把插图放在子目录里面![mhb] 02/08/10
\graphicspath{{figures/}{./}}

%%设置图形与标题的上下距离
\setlength{\abovecaptionskip}{0pt}
\setlength{\belowcaptionskip}{0pt}


%%设置图表标题格式
\usepackage[center]{caption}
\DeclareCaptionLabelSeparator{space}{~~}
\captionsetup{labelsep=space,labelfont={bf,small},textfont={small}}

%%公式相关
\usepackage{galois}


%%定理环境设置
\newtheorem{theorem}{\hspace{\parindent}定理}[chapter]
\newtheorem{corollary}{\hspace{\parindent}推论}[chapter]
\newtheorem{property}{\hspace{\parindent}性质}[chapter]
\newtheorem{lemma}{\hspace{\parindent}引理}[chapter]
\newtheorem{example}{\hspace{\parindent}例}[chapter]
\newtheorem{assumption}{\hspace{\parindent}假设}[chapter]
\newtheorem{problem}{\hspace{\parindent}问题}[chapter]
\newtheorem{definition}{\hspace{\parindent}定义}[chapter]
\newtheorem{remark}{\hspace{\parindent}注}[chapter]
\newtheorem{notation}{\hspace{\parindent}符号}[chapter]
% \newtheorem{proof}{}[chapter]
\newenvironment{proof}{\hspace{\parindent}\textbf{证明:}}{\hfill$\Box$}


\newcommand{\bn}{\begin{eqnarray}}
\newcommand{\ed}{\end{eqnarray}}
\newcommand{\bnn}{\begin{eqnarray*}}
	\newcommand{\edd}{\end{eqnarray*}}
\newcommand{\by}{\begin{array}}
	\newcommand{\ey}{\end{array}}
\newcommand{\nr}{\nonumber}
\newcommand{\lt}{\left}
\newcommand{\rt}{\right}
\newcommand{\de}{\stackrel{\triangle}{=}}
\def\lab{\label}
\def\hi{H_\infty}

\newcommand\old[1]{}

%%自定义列表环境
\newcommand{\namelistlabel}[1]{\mbox{#1}\hfill}
\newenvironment{namelist}[1]%
{\begin{list}{}
		{
			\let\makelabel\namelistlabel
			\settowidth{\labelwidth}{#1}
			\setlength{\leftmargin}{1.1\labelwidth}
			\setlength{\topsep}{0mm}
			\setlength{\itemsep}{0mm}
			\setlength{\parsep}{1mm}
		}
	}{\end{list}}

%%实现上标加括号的引用方式
\usepackage{hypernat}
\let\oldcite\cite
\renewcommand{\cite}[1]{\textsuperscript{\textsuperscript{\oldcite{#1}}}}
\let\citeayt\cite

%%重定义字号命令
\usepackage{calc}
\newlength{\tmpfontsizelength}
\newcommand{\selfontsize}[2][1]{%
	\setlength{\tmpfontsizelength}{#2*\real{#1}}%
	\fontsize{#2}{\tmpfontsizelength}%
	\selectfont%
}
\newcommand{\chuhaozi}{\selfontsize{42pt}}    % 初号, 1.5倍行距
\newcommand{\xiaochuzi}{\selfontsize{36pt}}    % 小初, 1.5倍行距
\newcommand{\yihaozi}{\selfontsize{26pt}}    % 一号, 1.4倍行距
\newcommand{\erhaozi}{\selfontsize{22pt}}    % 二号, 1.25倍行距
\newcommand{\xiaoerzi}{\selfontsize{18pt}}    % 小二, 单倍行距
\newcommand{\sanhaozi}{\selfontsize{16pt}}    % 三号, 1.5倍行距
\newcommand{\xiaosanzi}{\selfontsize{15pt}}    % 小三, 1.5倍行距
\newcommand{\sihaozi}{\selfontsize{14pt}}    % 四号, 1.5倍行距
\newcommand{\banxiaosizi}{\selfontsize{13pt}}    % 半小四, 1.5倍行距
\newcommand{\xiaosizi}{\selfontsize{12pt}}    % 小四, 1.5倍行距
\newcommand{\dawuzi}{\selfontsize{11pt}}    % 大五号, 单倍行距
\newcommand{\wuhaozi}{\selfontsize{10.5pt}}    % 五号, 单倍行距
\newcommand{\xiaowuzi}{\selfontsize{9pt}}    % 五号, 单倍行距
\newcommand{\liuhaozi}{\selfontsize{7.875pt}}    % 六号, 单倍行距
\newcommand{\qihaozi}{\selfontsize{5.25pt}}    % 七号, 单倍行距

%%脚注
\renewcommand{\thefootnote}{\fnsymbol{footnote}}

%%保存原章节命令
\let\oldchapter=\chapter
\let\oldsection=\section
\let\oldsubsection=\subsection
\let\oldsubsubsection=\subsubsection


\iffalse
%%设置章节编号类型
\def\thechapter{\arabic{chapter}}
\def\thesection{\thechapter.\arabic{section}}
\def\thesubsection{\thesection.\arabic{subsection}}
\def\thesubsubsection{\thesubsection.\arabic{subsubsection}}
\fi


%%重定义章节标题格式
\def\chaptername{第~\thechapter~章}
\def\sectionname{\thesection}
\def\subsectionname{\thesubsection}
\def\subsubsectionname{\thesubsubsection}


%%用titlesec宏包定义章节标题格式
\usepackage{titlesec}

%%设置章节标题格式
\def\chapterfmt{\centering\sanhaozi\heiti\bfseries}
\def\sectionfmt{\heiti\sihaozi\bfseries}
\def\subsectionfmt{\heiti\xiaosizi\bfseries}
\def\subsubsectionfmt{\heiti\xiaosizi\bfseries}


\titleformat{\chapter}[hang]{\vspace{-2cm}\chapterfmt}{\chaptername}{0.5em}{}[\vspace{-0.3cm}]
\titleformat{\section}[hang]{\vspace{-0.4cm}\sectionfmt}{\sectionname}{1em}{}
\titleformat{\subsection}[hang]{\subsectionfmt}{\subsectionname}{1em}{}
\titleformat{\subsubsection}[hang]{\subsubsectionfmt}{\subsubsectionname}{1em}{}

%%用titletoc宏包定义目录格式
\usepackage{titletoc}

%%设置目录里章节标题格式

\def\tocchapterfmt{\songti\xiaosizi}
\def\tocsectionfmt{\songti\xiaosizi}
\def\tocsubsectionfmt{\songti\xiaosizi}
\def\tocsubsubsectionfmt{\songti\xiaosizi}

\def\mydotfill{\hspace{.5em}\titlerule*[10pt]{$\cdot$}}
\titlecontents{chapter}[0em]{\tocchapterfmt} {\thecontentslabel\quad}{} {\mydotfill\contentspage}
\titlecontents{section}[2em]{\tocsectionfmt} {\thecontentslabel\quad}{} {\mydotfill\contentspage}
\titlecontents{subsection}[4em]{\tocsubsectionfmt} {\thecontentslabel\quad}{} {\mydotfill\contentspage}
\titlecontents{subsubsection}[4em]{\tocsubsubsectionfmt} {\thecontentslabel\quad}{} {\mydotfill\contentspage}






%%用tocbibind宏包自动将参考文献、索引、表格目录和图形目录等标题及其所在加入目录
%\usepackage{tocbibind}
\usepackage[nottoc]{tocbibind}

%%如果用ctexbook,设置相应章节标题格式
\ifx\CTEXsetup\undefined%
\else%
\CTEXsetup[name={第,章},number={~\arabic{chapter}~},format+={\chapterfmt}]{chapter}
% \CTEXsetup[name={},number={},format={\sectionfmt}]{section}
% \CTEXsetup[name={},number={},format={\subsectionfmt}]{subsection}
% \CTEXsetup[name={},number={},format={\subsubsectionfmt}]{subsubsection}
\fi


%%其它设置
%\setlength{\baselineskip}{22pt plus2pt minus1pt}
\setlength{\parindent}{12pt}
\setlength{\parskip}{1pt plus1pt minus2pt}
% \linespread{1.6}
\renewcommand{\baselinestretch}{1.50}

\renewcommand\arraystretch{1.00}

%%给pdf文件加入超链接
\usepackage{hyperref}
\hypersetup{
	CJKbookmarks=true,
	bookmarksnumbered=true,
	bookmarksopen=true,
	pdfstartview=FitH,
	bookmarks=true,
	linkbordercolor={1 1 1},
	colorlinks=true,
	filecolor=blue,
	linkcolor=black,
	citecolor=blue,
	anchorcolor=green,
	urlcolor=blue
}

\usepackage{bm}
\usepackage{algorithm}
\usepackage{algorithmic}

\makeatother